\documentclass[../chapter1/thesis_msc.tex]{subfiles}

\begin{document}
    \chapter*{Abstract}
    \addcontentsline{toc}{chapter}{Abstract}
    \phantomsection
    \thispagestyle{plain}
\paragraph{}Pulsars are remnants of intermediate mass stars (8-25 $M_\odot$) that emits electromagnetic radiation modulated by the spin of the pulsars, mostly detected at radio frequencies. Pulsars are used as tools to study many physical phenomena, such as mapping the distribution of interstellar medium in the Galaxy. The simplest pulsar search technique involves searching for bright single pulses in the time series of amplitudes recorded from a specific location on the sky. However, most of the pulsars are too weak to detect as a single pulse. As a result, the majority of pulsars have been discovered by applying The Fast Fourier Transform (FFT) to a time series and searching for periodicities in the Fourier spectrum.
However, the FFT has some limitations, for example, detecting long-period pulsars (period  $> \sim$1s). This limitation might cause us to miss some long period pulsar from the previous pulsar survey which typically used FFT.
An alternative way to search for periodicities is called ``The Fast Folding Algorithm" (FFA). The FFA creates folded pulse profile (amplitude-phase plot) for each trial period as then evaluated to determine if a pulsar is present. 
 I demonstrated the analytic comparison between FFT, FFA and single-pulse search efficacy shows that FFA is better than FFT when the pulse period is longer than  $\sim$ 1s. However, if the period of the pulsar is long enough to detect for just a few pulses, the single pulse search outperforms the FFA.  

\paragraph{} Radio frequency interference (RFI) affects the detectability of pulsars by creating numerous artificial pulsar candidates. 
The mechanism for identifying and reducing RFI is called RFI mitigation. In this work, I shall present a new method of RFI mitigation and apply it to over 70 million pulsar candidates detected in the HTRU-N survey FFT pipeline using Effelsberg telescope to identify periodically static RFI. This new method shows that the Effelsberg telescope has both dynamic and static RFI. Moreover, lists of periodic RFI for each observation day are generated and ready to be used to reduce the number of artificial candidates from RFI during reprocessing of HTRU-N.  

\paragraph{} Pulsars in binary systems are affected by binary motion, causing a Doppler shift which changes in their apparent period smearing and bedim pulse profiles. 
A straightforward way to reduce this period change, called `acceleration searching', which consists of assuming a constant acceleration and re-sampling the time series with each acceleration step. I created acceleration-FFA pipeline for this thesis as well as an optimisation of the acceleration step size. The calculation of the FFA acceleration step shows that the FFA is highly optimised for searching periods range which reduces unnecessary trial, making the pipeline faster. A test of the acceleration search using the FFA shows that for a simulated $\sim$ 5600 pulsars with periods ranging from 1.0s to 6.0s in binary systems with companions mass of 0-37 $M_\odot$, the pipeline detected  90 \% of the pulsar with less than 10\% reduction  compare to an equivalence solitary pulsar.  

\paragraph{} Using the acceleration-FFA pipeline, I have processed 122 observations from the HTRU-S low-lat observed with Parkes telescope.
Three new pulsar candidates have been discovered from this pipeline, one of them has been confirmed to have a spin period of 1.89s (PSR J1746-28). 
Our method also redetect a pulsar that had been overlooked previously by the FFT pipeline with a period of $\approx$ 0.7s. Moreover, 54 known pulsars with periods range of $\approx$ 0.14s to 6.0s were detected in this pipeline. Studying the distribution of 54 known pulsars for brightness and period suggests that they are in agreement with the analytic comparison between FFT and FFA. I also confirmed a FAST telescope pulsar candidate with a period of $\approx$ 4.9s using three observations from Effelsberg telescope. 
    
    \clearpage
    \thispagestyle{empty}
\end{document}
