\documentclass[../chapter1/thesis_msc.tex]{subfiles}
%\paragraph{} The comparison of various pulsar search method shows that the FFT is effective to search for short period pulsar. The FFA is effective for long period pulsars and single pulse searching is effective to search for pulsar with a few pulses.
\begin{document}
\chapter{Conclusions and Future work} \label{Con}
\paragraph{} This work is divided into two parts. First, I talked about a new RFI mitigation method. This method takes an advantage of the fact that most of the pulsar candidates are RFI, decreasing the possibility of finding pulsars. Because of this, the study of the distribution of pulsar candidates in the period domain allows us to understand the distribution of RFI. The second part is about using the FFA to discover the missing long period pulsars in the HTRU-S low lat. The overall conclusion in this work will be mentioned here. 



\section{FFA pipeline}
\paragraph{} As mentioned in Chapter \ref{Methods}, the FFT pipeline has discovered most of the known pulsars despite its limitation of searching for long period pulsars. In this work, an alternative way to search for pulsars, the FFA, is used. The data is taken from the HTRU-S low lat pulsar survey which has the longest observation time in the HTRU survey concentrate on low Galactic longitude. Long observation time allows more pulses found in the data which are important for detecting long period pulsars.

\subsection{The Acceleration FFA (AFFA) pipeline}
\paragraph{} This pipeline is aiming to search for long period and dim pulsars in the Galactic plane which contains most of the stellar populations. Also, known pulsars are needed to be re-detected to compare this pipeline to the FFT pipeline. At least 10 pulses are assumed to be the proper amount of pulses per observation that is enough to detect the pulsar significantly. As a result, the longest search period in this work is 430s which is 10 \% of the observation time. The FFA is in inefficient for using to search for short period pulsars. As a result, the shortest period needs to be optimised otherwise the pipeline will take too long to search for pulsars. In this thesis, the searching time is optimised to be $\sim 12$ hours which is corresponding to the minimum search period of 0.131s. The DM range is from 0 to 3072 $pc~cm^{-3}$ while the acceleration range is -128 to 128 $m/s^2$. This acceleration range is corresponds to the maximum acceleration produced from the system of pulsar with a companion mass of 37 $M_\odot$. This acceleration is already bigger than the full observation acceleration ranges in the FFT pipeline, which is $^+_-1 ~m/s^2$ for the full observation \citep{ng2015high}. This allows us to search for heavier pulsars in binary systems than the previous pipeline. The DM and acceleration step for searching period range from 0.131s to 0.262s are 3 $pc~cm^{-3}$ and 1 $m/s^2$ respectively. Since the FFA is searching for a range of periods in unit of bins, when the searching period is two times longer, the down sampling is applied with a factor of two. As the sampling time is getting twice longer, the search period range, the acceleration step, and DM step are also getting twice bigger, reducing the number of operations. This process applies iteratively until the maximum search period reaches more than 430s. Noted that, this acceleration search is effective only for the systems with an orbital period more than 12 hours otherwise the constant acceleration will not be valid. The pipeline used the Hercules cluster in Garching, Germany as the main computing facility. The clusters have 184 processing notes with 24 processing cores per node \footnote{ Details regarding the Hercules cluster can be found at https://www.mpcdf.mpg.de/services/computing/linux/radioastronomy}. 
\subsection{Simulation test of the AFFA}
\paragraph{} To test the S/N loss from the acceleration re-sampling part of the pipeline which is implemented in this work, the 5600 simulated pulsars in binary systems are applied to the FFA pipeline. The simulated pulsars have spin periods randomly injected between 1.0s to 6.0s, with pulse duty cycles $\delta$ randomly ranging from 1\% to 25\%. After that, I also generated pulsar with the same P and $\delta$ but with companion mass randomly generated between 0 to 37.0 $M_\odot$ at an orbital phase of 0.25 which is the phase that gives maximum and constant acceleration and an orbital period of 12 hours. The detected S/N ratio between simulated pulsars and themselves isolated version are compared. The result shows that only 10\% of simulated pulsars have more than 5\% S/N loss. This 10\% of pulsars has narrow pulse with duty cycle narrower than 6.25 \%, corresponding to 8 profile bins from 128 bins. This error is due to the coding mistakes that allows the pulse to smear more than 8 bins. This work shows that the AFFA pipeline can recover most of the sensitivity with improvements in the future. 
\subsection{Results}
\paragraph{} Applying 122 observations from the HTRU-S low lat to the AFFA pipelines gives us re-detection of 41 known pulsars. These known pulsars have periods ranging from 190ms to 6s.  The detected S/N ratio in the AFFA pipeline is compared to the simple FFT pipeline. Results show that the S/N from the AFFA pipeline is more than the S/N from the FFT pipeline when the period is getting longer. There are some unexpected results for pulsars with periods longer than 1s but have higher S/N in the FFT pipeline which still need more investigations. Even the data that was used in this work is relatively small compared to the full HTRU-S low lat data ($\sim$ 1\% of the survey). This pipeline detected one new confirmed pulsar, two pulsar candidates, and one known pulsar that was missed in the FFT pipeline. 

\paragraph{} The new pulsar candidate was found approximately in 0.5$^o$ from the Galactic centre with a period of $\sim$1.9s and DM of 1290 $pc~cm^{-3}$ with relatively weak S/N of 11.2. This observation took place in May 2011. First, the three observations from the PMPS that covered the same position in the sky are used to confirm these candidates by folding with the same period and DM. However, none of the observations give significant detection of this pulsar. This might be caused by the fact that the observation time in the PMPS is shorter than the HTRU-S low lat. This means that fewer pulses are in these observations. Fewer pulses imply lower S/N which cause the pulsars to be detected in the S/N below the significant threshold.  An observation at the same position in the sky is performed using the Parkes telescope with the same length of observation in April 2018 and the AFFA pipeline is applied to search for a signal with a similar period and DM. The result of the follow up shows a detection at approximately the same period and DM. This re-observation confirmed that this candidate is a real pulsar because the detection is found in the same position in the sky eventhough the observation time is  almost seven years apart. After that, the timing campaign for this pulsar was begun to get full timing solutions, i.e. period, DM, and period derivative. The period and period derivative are used to locate the position of these pulsars on the $P-\dot{P}$ diagram. Also, two new pulsar candidates are detected with periods of 0.523s and 1.229s. Two pulsar candidates will be followed up in the future using both the PMPS data and re-observation using the Parkes telescope with the same observation length.
\paragraph{} Not only a new pulsar but also one known pulsar (J1759-1903) that was missed in the HTRU-S low lat the FFT pipeline were found. This pulsar has a period of 0.7315s and DM of 430 $pc~cm^{-3}$. Applying the FFT pipeline created by \cite{Ng} allows the detection of this pulsar in the FFT pipeline with lower S/N. This might be due to a human error. Noted that, this pulsar is considered to have relatively short period, which means that the FFA pipeline is still better than the FFT pipeline even in the relatively short period.
\paragraph{} The AFFA pipeline is applied not only to the HTRU-S low lat data but also to the Effelsberg telescope data to confirm a pulsar candidate from the FAST telescope pulsar survey. This candidate has a period around 5s and polluted by the RFI in the normal FFT pipeline. However, this pulsar is detected clearly in the AFFA pipeline. This shows that this pipeline is promising in order to use for searching for a pulsar in the other data sets.

\section{Future work}
\subsection{Survey processing}
\paragraph{} As only $\sim$ 1 \% of the HTRU-S low lat data is applied to the AFFA pipeline, the pipeline takes 12 hours per observation for a compute node. Unfortunately, just $\sim$ 1 \% of the observations have been processed which is not enough to predict the number of pulsars that were missed by the FFT pipeline. As the Hercules cluster has 184 compute nodes and the HTRU-S low lat has 15600 observations, the expected computational time for all of the HTRU-S low lat survey using the full capacity of Hercules cluster will take only 42 days to complete the whole the HTRU-S low lat. However, since the number of candidates per observation is around 1000, the number results will be approximately 7.8 million which is too much for a human to inspect manually. To overcome this, a machine learning algorithm is required to reduce the number of pulsar candidates. In addition, the improved RFI mitigation is also reducing the number of artificial pulsar candidates.  As demonstrated before, the AFFA pipeline can be adapted to other surveys such as the HTRU-N and the FAST pulsar survey. Applying this pipeline to other surveys will definitely help to detect more pulsars in the future. %Moreover, the AFFA pipeline is optimised to search for a range of parameters, i.e. period, DM, and accelerations. As a result, the AFFA is ideal to search for systems that we have some parameter assumptions.
%\paragraph{} Used RFI mitigation
%\paragraph{} Search parameters 
\paragraph{} Newly discovered pulsars are needed to be observed over a long time span to get the timing solution which is the list of parameters to coherently fold pulsars. In this work, only J1746-28 has been observed regularly since April 2018. However, the time spans for these observation are not yet long enough to get other parameters except the period and DM. For pulsar candidates including candidates in the future, candidates will be mainly followed up using the data from the PMPS pulsar survey if the data is available. However, if the candidates are not detected in the PMPS, other available telescopes will be used to observe instead. After confirmations of candidates, the candidates will be observed regularly to get the full parameters, especially $\dot{P}$. Measuring of the period and $\dot{P}$ for larger number of pulsars discovered in this pipeline will answer the question of whether the population of pulsars that the AFFA pipeline will detect will be different from the FFT pipeline's pulsars.   
%\subsection{Ongoing pulsar timing and additional follow-up}
\subsection{Developing the AFFA pipeline}
\paragraph{} The AFFA pipeline shows a potential to detect pulsars with long period. However, the pipeline still needs to be improved. Firstly, a coding mistake makes the pipeline less sensitive to pulsars with $\delta$ smaller than 6.25 \%. This coding mistakes have been corrected. After corrected the mistakes, the pipeline is expected to be more sensitive for even pulsars with  $\delta$ more than 0.78 \% (1/128 bins).  This pipeline has the limitation on acceleration search, which assumed that the accelerations are constant for the whole observation. This assumption limits the pipeline to search for pulsar in the binary system with long orbital period and at the point where the acceleration is constant. The `jerk'' ($\frac{da}{dt}$) search can compromise this limitation by searching for acceleration and jerk. This can improve our change to find a new pulsar in the binary system. Especially the pulsar-BH binary which is expected to be a slow pulsar due to the pulsar evolution.      
%\paragraph{} As mentioned before, the AFFA pipeline is effective to search for slow pulsars with high mass companion Moreover, the FFA is searching for pulse profiles with match filter. So, 
%\paragraph{} Template matching

%\subsection{Potential of discovery}
%\paragraph{} Slows pulsars 
%\paragraph{} Magnetars 
%paragraph{} FRBs 
%\paragraph{} Binary system with high mass companion. 

\section{Final remarks}
\paragraph{} As described earlier in this work, the FFA algorithm has a potential to minimise the disadvantage of the FFA algorithm to search for the long period pulsars. Moreover, the FFA shows a potential for highly optimisation for acceleration search. As a conclusion, the FFA algorithm could discover the missing slowest pulsars or slow pulsars in high mass binary systems in the future.  
%\section{Citations}


\end{document}