\documentclass[../chapter1/thesis_msc.tex]{subfiles}
%\paragraph{} The comparison of various pulsar search method shows that the FFT is effective to search for short period pulsar. The FFA is effective for long period pulsars and single pulse searching is effective to search for pulsar with a few pulses.
\begin{document}
\chapter{Conclusions and Future work} \label{Con}
\paragraph{} This work is divided into two parts. First, I talked about new RFI mitigation method. This method takes an advantage of the fact that most of the pulsar candidates are RFI increasing the possibility of finding pulsars. Then, study the distribution of pulsar candidates in the period domain allows us to understand the distribution of RFI. The second part is about using FFA to discover missing long period pulsar in the HTRU-S low-lat. The overall conclusion in this work will be mentioned here. 
\section{New RFI mitigation method}
\paragraph{} In this work, the pulsar candidates from the HTRU-N survey using he FFT pipeline were used. The data span from July 2011 to December 2014. The distribution of RFI over the period is studied by making histograms of the number of pulsar candidates over periods. To avoid the bias on the number of observations on each day, the histogram is normalised by the number of candidates on that day.
\paragraph{} To study the static RFI, the ``'RFI profile' is created from the mean histogram of the whole data set and subtract that from the data. This method can also can 
find dynamics RFI. However, residual histogram still not able to be used to identify the RFI since there is still too much RFI to flag otherwise the real data might be flagged as well. The alternative method that identifies RFI is to study the number distribution of RFI in each day and flag period bins that not follow Gaussian distribution recursively. The result shows that the dynamics and static RFI is identified. Only 5 \% of all the period bin is flagged as RFI which means only fews period bins is flagged. The list of flagged periods is converted to zap table ready to be used in the HTRU-N reprocessing. This ``zap list'' is used as the list of period bins that are polluted by periodic RFI. Ignoring these period bins speeds up the process of candidates reviewing.  


\section{FFA pipeline}
\paragraph{} As mentioned in chapter ~\ref{FFA_c}, the FFT pipeline has discovered most of the known pulsars despite its limitation at searching for long period pulsars. In this work, an alternative way to search for pulsars the FFA is used. The data is taken from the HTRU-S low-lat pulsar survey which is the longest observation time in HTRU survey concentrate on low Galactic longitude. Long observation time allows more pulses which are important for detecting long period pulsars.  
\subsection{The Acceleration FFA (AFFA) pipeline}
\paragraph{} This pipeline is aiming to search for long period and dim pulsars in the Galactic plane which contains most of the stellar populations. Also known pulsars needs to be re-detected to compare this pipeline to the FFT pipeline. At least 10 pulses is assumed to be proper amount of pulses per observation that is enough to detect the pulsar significantly. As a result, the longest search period in this work is 430s which is 10 \% of the observation time. The FFA is in inefficient to search for short period pulsars. As a result, the shortest period needs to be optimised othewise the pipeline will take too long to search for pulsars. In this thesis, the searching time is optimised to be $~\sim 12$ hours which is corresponding to the minimum search period of 0.131s. The DM range is from 0 to 3072 $pc~cm^{-3}$ while the acceleration range is -128 to 128 $m/s^2$. This acceleration range is corresponds to the maximum acceleration produced from the system of pulsar with a companion mass of 37 $M_\odot$. This acceleration is already bigger than the full observation acceleration ranges in the FFT pipeline which is $^+_-1 ~m/s^2$ for the full observation ~\citep{ng2015high} allows us to search for heavier pulsars in binary systems than previous pipeline. The DM and acceleration step for searching period range of 0.131s to 0.262s is 3 $pc~cm^{-3}$ and 1 $m/s^2$. Since the FFA is searching for a range of periods in unit of bins, when the searching period is two times longer, the downsampling is applied with a factor of two. As the sampling time is getting twice longer, the search period range, the acceleration step and DM step getting twice bigger, reducing the number of operations. This process applies iteratively until the maximum search period reaches 430s. Note that, this acceleration search is effective only for the systems with an orbital period more than 12 hours otherwise the constant acceleration will not be valid. The pipeline used the Hercules cluster in Garching, Germany as the main computing facility. The cluster have 184 processing note with 24 processing core per node \footnote{ Details about the Hercules cluster can be found at https://www.mpcdf.mpg.de/services/computing/linux/radioastronomy}. 
\subsection{Simulation test of the AFFA}
\paragraph{} To test the S/N loss from the acceleration re-sampling part of the pipeline which is implemented in this work. The 5600 simulated pulsars in binary systems are applied to the FFA pipeline. The simulated pulsars have spin periods randomly injected between 1.0s to 6.0s, with pulse duty cycles $\delta$ randomly ranging from 1\% to 25\%. After that, we also generated pulsar with the same P and $\delta$ but with companion mass randomly generated between 0 to 37.0 $M_\odot$ at an orbital phase of 0.25 which is the phase that gives maximum and constant acceleration and an orbital period of 12 hours. The detected S/N ratio between simulated pulsars and itself isolated version is compared. The result shows that only 10\% of simulated pulsars have more than 5\% S/N loss. This 10\% of pulsars has narrow pulse with duty cycle narrower than 6.25 \%, corresponding to 8 profile bins from 128 bins. This error is due to the coding mistakes that allow the pulse to smeared more than 8 bins. This work shows that the AFFA pipeline can recover most of the sensitivity with improvements in the future. 
\subsection{Results}
\paragraph{} Applying 122 observations from HTRU-S low-lat to the AFFA pipelines gives us re-detection of 41 known pulsars. Theses know pulsars have periods range from 190ms to 6s.  The detected S/N ratio in the AFFA pipeline is compared to the simple FFT pipeline. Results show that the S/N from the AFFA pipeline is more than the S/N from FFT pipeline when the period is getting longer. There are some unexpected results for pulsars with periods longer than 1s but have higher S/N in the FFT pipeline which still needs more investigations. Even the data that was used in this work is relatively small compared to full HTRU-S low-lat data ($~\sim$ 1\% of the survey), this pipeline detected one new confirmed pulsars, two pulsar candidates and one known pulsar that missed in the FFT pipeline. 
\paragraph{} The new pulsar candidate found approximately in 0.5$^o$ from the Galactic centre with a period of 1.9s and DM of 1290 $pc~cm^{-3}$ with relatively weak S/N of 11.2. This observation took place in May 2011. First, the three observations from PMPS that covered the same position in the sky are used to confirm these candidates by folding with the same period and DM. However, none of the observations give significant detection of this pulsar. This might cause by the observation time in PMPS is shorter than HTRU-S low-lat which means fewer pulses in these observations. Fewer pulses imply lower S/N which might cause the pulsars to be detected in the S/N below the significant threshold.  An observation at the same position on the sky is made using the Parkes telescope with the same length of observation in April 2018 and apply the AFFA pipeline to search for a signal with a similar period and DM. The result of follow up shows a detection at approximately the same period and DM. This re-observation confirmed that this candidate is a real pulsar because the detection in the same position on the sky even though the observation time are almost seven years apart. After that, the timing campaign for this pulsar was begun to get full timing solutions, i.e. period, DM, period derivative. The period and period derivative are used to locate the position of these pulsars on the $P-\Dot{P}$ diagram. Also, two new pulsars candidates are detected with periods of 0.523s and 1.229s respectively. Two pulsars candidates will be followed up in the future using both PMPS data and re-observation using Parkes telescope using the same observation length.
\paragraph{} Not only a new pulsar but also one known pulsar (J1759-1903) that were missing in HTRU-S low-lat the FFT pipeline was found. This pulsar has period of 0.7315s and DM of 430 $pc~cm^{-3}$. Applying the FFT pipeline created by ~\cite{Ng} gives the detection of this pulsar in the FFT pipeline with lower S/N. This might cause from the outnumber of pulsar candidates from RFI. Note that, this pulsar is consider to has relatively short period which means that the FFA pipeline still better than the FFT pipeline even in the relatively short period.
\paragraph{} The AFFA pipeline is applied not only to HTRU-S low-lat data but also the Effelsberg telescope data to confirmed a pulsar candidate from the FAST telescope pulsar survey. This candidate has a period around 5s and polluted by the RFI in normal FFT pipeline but it is detected clearly in the AFFA pipeline. This shows that this pipeline is promising to search for a pulsar in the others data set.

\section{Future work}
\subsection{Survey processing}
\paragraph{} As only $~\sim$ 1 \% of HTRU-S low-lat data applied to the AFFA pipeline. The pipeline takes 12 hours per observation for a compute node. Unfortunately, just $~\sim$ 1 \% of the observations have been processed which is not enough to predict the number of pulsars that were missed by the FFT pipeline. As the Hercules cluster has 184 compute nodes and HTRU-S low-lat has 15600 observations, the expected computational time for all of the HTRU-S low-lat survey using the full capacity of Hercules cluster will take only 42 days to completes the whole HTRU-S low-lat. However, since the number of candidates per observation are around 1000, the number results will be approximately 7.8 million which is too much for a human to inspected by eye. To overcome this, a machine learning algorithm is required to reduce the number of pulsar candidates. In addition, improving of the RFI mitigation is also reducing the number of artificial pulsar candidates.  As demonstrated before that the AFFA pipeline can be adapted to others surveys such as HTRU-N and FAST pulsar survey. Applying this pipeline to other survey will definitely going to detect more pulsars in the future. %Moreover, the AFFA pipeline is optimised to search for a range of parameters, i.e. period, DM, and accelerations. As a result, the AFFA is ideal to search for systems that we have some parameter assumptions.
%\paragraph{} Used RFI mitigation
%\paragraph{} Search parameters 
\paragraph{} Newly discovered pulsars need to be observed over a long time span to get the timing solutions which is the list of parameters to coherently fold pulsars. In this work, only J1746-28 has been observing regularly since April 2018. However, the time spans for these observation is not yet long enough to get others parameters except period and DM. For pulsar candidates including candidates in the future, candidates will be mainly followed up using the data from the PMPS pulsar survey if the data is available. However, if the candidates are not detected in the PMPS, other available telescopes will observe instead. After confirmations of candidates, the candidate will be observed regularly to get full parameters especially $\Dot{P}$. Measuring of period and $\Dot{P}$ for larges number of pulsars discovered in this pipeline will answers the question whether the populations of pulsar that the AFFA pipeline will detect will be different from the FFT pipeline's pulsars.   
%\subsection{Ongoing pulsar timing and additional follow-up}
\subsection{Developing the AFFA pipeline}
\paragraph{} Even the AFFA pipeline shows a potential to detect pulsars with long period. The pipeline still be able to improved. Firstly, as a coding mistakes makes the pipeline less sensitive to pulsars with $\delta$ smaller than 6.25 \%. This coding mistakes has been corrected. After corrected the mistakes, the pipeline is expected to be more sensitive for even pulsars with  $\delta$ more than 0.78 \%.  Moreover, to reduce the limitation of acceleration search which assumed that the accelerations are constant for the whole observations. This assumption limited the pipeline to search for pulsar in binary system with long orbital period and at the point where the acceleration is constant. The ``jerk'' ($\frac{da}{dt}$) search can compromised this limitation by searching for acceleration and jerk. This can improves our change to find a new pulsar in binary system. Especially the pulsar-BH binary which is expected to be slows pulsar due to  the pulsar evolution.      
%\paragraph{} As mentioned before, the AFFA pipeline is effective to search for slow pulsars with high mass companion Moreover, the FFA is searching for pulse profiles with match filter. So, 
%\paragraph{} Template matching

%\subsection{Potential of discovery}
%\paragraph{} Slows pulsars 
%\paragraph{} Magnetars 
%paragraph{} FRBs 
%\paragraph{} Binary system with high mass companion. 

\section{Final remarks}
\paragraph{} As described in this work, the FFA algorithm has a potential to compromise the disadvantage of the FFA algorithm to search for the long period pulsars. Moreover, the FFA shows a potential for highly optimisation for acceleration search. As a conclusions, the FFA algorithm could discovered the missing slowest pulsars or slow pulsars in high mass binary systems in the future.  
%\section{Citations}


\end{document}